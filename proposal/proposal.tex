\documentclass[a4paper,10pt,notitlepage]{article}
\usepackage[utf8]{inputenc}
\usepackage{mathtools}
\usepackage{amsmath}
\usepackage{graphics}
\usepackage{epsfig}
\usepackage{multicol}
\usepackage{tabularx}
%\usepackage{breakcites}
\usepackage[breaklinks]{hyperref}
\usepackage{listings}
\usepackage{fullpage}

\lstset{
	breaklines = true
	}


\title{Biological Databases Final Project Proposal}
\author{
    \begin{tabular}{l l}
    Students: & Amy Li, Michelle, Ben Pote \\
    Supervisor: & Ruslan Afasizhev \\
    Contact: & 617-414-1056 \\
    & ruslana@bu.edu
\end{tabular}
}


\begin{document}
\lstset{language=R}

\maketitle

\begin{section}{Project Overview}
Trypansoma brucei (T. brucei) is a single cell protist parasite that gives rise to the African trypanosomiasis, 
or sleeping sickness. The mitochondrial genome of T. brucei consists of a network of circular structures called 
minicircles (~1kb) and maxicircles (10kb). Maxicircles encode for protein-coding genes. However, such transcripts
can be translated only after posttranscriptional deletion and insertion of uridines through a process called 
RNA-editing. This editing process is directed by short stretches of guideRNAs, which are encoded in minicircles.
Understanding the organizational structures of minicircles and its encoded guideRNAs is an important step for 
deciphering the RNA-editing mechanism in T. brucei. 

Here, we create a database for visualising next-generation sequencing data from mitochondrial minicircle DNA and
small RNAs (putative guideRNAs) isolated from T. brucei. The database will facilitate the storage and analysis of 
minicircle sequences. Analyses to be performed include aligning reads and categorizing minicircles based on the 
minicircle conserved regions they contain.

Using MSAD, users would be able to retrieve and download information on minicircle sequences based on specific 
filtering criteria, including presence of conserved regions (CSBs), alignment coverage by smallRNA, etc. In 
addition, we wish to provide visualization for the minicircle sequence alignment information such as displaying 
the pileup of mapped sRNA reads along a specified minicircle query. If time permits, we will also visualize 
alignment for a given minicircle cluster.
 
With this database, users will be able to answer questions such as:

\begin{enumerate}
    \item Which putative minicircle sequences contain conserved regions (CSB1, CSB2, CSB3)?
    \item What is the multiple sequence alignment (MSA) for the specific cluster of sequences?
    \item For a given minicircle sequence, which regions have high coverage of mapped sRNA reads? Where are the CSB regions (if any)?
\end{enumerate}


Upon completion of the project, the database will be made accessible to the Afasizhev lab and BU affiliated 
students working on this dataset. If successful in this initial deployment, there is a possibility that the 
database could be made publicly available for other researchers to use.


Our database will contain data from the following:

\begin{enumerate}
\item PacBio Minicircle Sequencing Dataset - this dataset contains 39,939 filtered minicircle DNA sequences isolated from the mitochondria of Trypanosoma brucei. These reads are on average around 1kb long. For each sequence, we will have a unique sequence identifier, the actual DNA sequence, and a cluster assignment. The cluster assignment has already been precomputed using Connected Component Clustering, based on sequence similarity. We are currently in the process of applying other clustering methods, so we are interested in adding additional cluster assignments later on.
\item Illumina smallRNA-seq dataset - this contains 3,937,040 unique processed smallRNA reads which are putative guideRNAs. Each read has a duplication number (how many duplicate sequences of itself is found in the original sequence file), an RNA sequence, and alignment information: where each sRNA read aligned to each of the minicircle sequences (some align to more than 1 or no minicircle sequences, and can align at more than two different positions within the same minicircle sequence). This is a very big file, so we may just store alignment information (processed from BED format) into the database and not each individual smallRNA sequence.
\item The sequences for the known minicircle conserved regions CSB1, CSB2, and CSB3. These are short sequences that are commonly found in curated minicircles (CSB3 is associated with the origin of replication). We’d like to be able to query which putative minicircle sequences contain each or combinations of these regions. 
    \end{enumerate}

Multiple Sequence Alignments: given a cluster id, we can retrieve minicircles associated with such cluster and 
generate a Multiple Sequence Alignment of these sequences on the fly (or precomputed).


\end{section}

\begin{section}{ER diagram}
    Here we present an entity-relationship diagram that represents how we plan to build our database:

    \includegraphics[width=.8\textwidth]{ERdiagram-crop.pdf}
    
\end{section}


\end{document}
